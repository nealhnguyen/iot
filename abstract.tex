IoT devices are one of the fastest growing class of devices in today's technology-driven world. However, with so many new internet connected devices, there is little knowledge of what they are doing and whether or not these devices are genuinely secure.  There needs to be more data and analysis on IoT devices.

To better understand IoT devices, we have built an expansive database with network and power data for 16 IoT devices over ten months accumulating 184.94 GB of data and 172,445,929 rows of data that comes with a tool that graphs the database's data for analyis. We have documented the creation, logging, and utilization of this database within this paper.

From this database, we were able to classify smart speakers from just one minute of power data, even with noise from various high power devices. We also found interesting tradeoffs that smart speakers made between power and network usage, startup patterns in streaming devices, and other patterns in our IoT testbed.

It has shown us that there is a lot to learn about IoT devices from the network usage. However, there is also a lot to learn from power information. If its possible to classify a device in use from just the power, with some noise, it is possible someone could figure out what is going on in a household from just monitoring their powerline. We intend to increase interest in the analysis of these devices and provide an expansive database for other researchers to further understand IoT devices.