\chapter{Discussion}
\label{Discussion}
In this chapter, we discuss the implications of the data presented in chapter \ref{Results}.

\section{Switching Smart Plug}
In section \ref{swappingSwitch}, we displayed figures \ref{fig:swapEcho1Home}, \ref{fig:swapEufy1Echo1}, and \ref{fig:swapEufy1Home}. We do this to ensure consistency between measurements. The network traffic is assumed to be consistent because all network traffic goes through our NUC AP, following a standard WiFi connection. However, each device has their power plug in different locations on different power strips. We wanted to check to see that this variation causes minimal recording differences. From the figures \ref{fig:swapEcho1Home}, \ref{fig:swapEufy1Echo1}, and \ref{fig:swapEufy1Home}, we can see that after restarting, the power level goes back to where it was before. We did this multiple times with other switches to different locations and obtained nominal changes in the idle power usage. Thus, we can conclude that this power logging scheme is relatively consistent.

\section{Metadata Analysis}
In sections \ref{wholeDB}, \ref{smartSpeakerResults}, and \ref{Streaming Devices}, we go over the protocol spread, and general power/network usage of varying devices. However, because Frawley's paper \cite{frawley_2018} already analyzes these sections, refer to his paper for analysis.

\section{Power Spike at 18:12}
In figure \ref{fig:bestBballSum}, we saw an unaccounted power spike at 18:12. When looking at figures \ref{fig:bestBballSeperate} and \ref{fig:bestBballNetwork}, we believe that the power surge is from the LEDs. Because there is no network usage during this time, we do not think the Echo Dot was doing anything data recording, listening, or was activated. We believe that the computer was covering the Echo Dot, but is briefly moved, exposing the Echo Dot to more light, thus causing the LEDs to brighten.

\section{Device Fingerprinting}
We want to see if it is possible to determine what devices are in use if given access to home's power line. Classifying devices from a single power usage graph over time consisting of multiple devices is a good simulation for a total powerline.

In sections \ref{sumPowerGraph} and \ref{sumPowerGraphWithNoise}, we show how the smart speakers perform under various operations such as ``what's the weather'', ``who is the best basketball player'', or ``what's the news''. We did this various times to confirm consistency, and we always found that there is a start spike (voltage spike peak to peak value) of around 390 mW with 36.1 mW standard deviation for the Eufy Genie, 606.7 mW spike with 110 mW standard deviation for the Google Home, and 2026.7 mW spike with 149.89 mW standard deviation for the Echo Dot as shown in figure \ref{fig:spikeVoltages}. From the graph, we can determine what device model is being used if it is within those thresholds. The models with multiple units are consistent with each other. From this, we conclude it is possible to determine a device model from visual examination of its power usage. This is our first step to testing our hypotheses that it is possible to see what devices are in use from analysis of someone's power line.

But in a real house, there are more than just 5 smart speakers on a power line. The next step is to add noise from high power devices to see if it is still possible to visually determine the device in use from power spikes \ref{sumPowerGraphWithNoise}. From the graphs shown, if the power is stable enough, then it is not a problem for classifying the devices. The device spikes are shifted up. This statement is supported by our graphs with the fan, the idle microwave, idle fridge, and idle NUC.

However, if the power usage has noise on the same magnitude or larger than the smart speaker spikes, then it becomes difficult to determine which smart speaker model is currently in use. This is evident when the NUC, fridge, and microwave are in use. However, the microwave and fridge are idle for most the time, so it is still possible to monitor someone's power line on a time frame that these devices are not being used. However, PCs, smartphones, and streaming devices are constantly in use, it would be difficult to find a time one the NUC is not in use and make it challenging to determine what smart speaker model is in use from just visual analysis of its power usage.

From the results of section \ref{sumPowerGraph} and \ref{sumPowerGraphWithNoise}, it is possible to classify the model of a smart speaker from visual examination of a single power usage graph during a time when most other devices are idle. However, with multi-use devices such as a PC, smartphone, or IoT device that is in constant use for long periods of times, it would be challenging to classify a smart speaker model.

\section{Smart Speaker Comparison}
Finally, we noticed that the Echo Dot is characterized by a power spike that is a lot larger than the other smart speakers and wer interested to see what trade-offs these devices made.

From the results in section \ref{smartSpeakerComparisonSection}, we can see that the echo uses the most power, then the Eufy, then the Google Home. However, the Eufy uses the most network throughput, then the Google Home, then the Echo Dot. We believe the Eufy is constantly communicating through the network so that it can cache results, minimizing onboard operations. The Echo Dot, on the other hand, uses the least network throughput and opts to do as much on board work as possible. The Google Home sits in the middle, querying the network, for more relevant things that it can cache for when the user asks the questions. It seems they pre-request results better. It is interesting that the Google Home still sits in the middle for network throughput when it is the only one that sends UPnP at such a high frequency. However, this is mostly speculation and would be interesting to see more research on this.