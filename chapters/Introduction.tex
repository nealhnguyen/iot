\chapter{Introduction}
\label{Introduction}
In today's world, there are 23.14 billion IoT devices worldwide with an expected 75.44 billion IoT devices in 2025 \cite{statista_2016}. This rush of IoT devices has ushered in an industrial era for IOT devices. Should we be concerned about the data sent to servers? Are the IoT devices in our home spying on us?

We believe so. The enormous demand, production numbers, and production time can lead to many security flaws and little analysis of this new class of devices. For example, Amazon stated that the echo dot listens to snippets of audio, and only keeps it if the wake word is present. However, in the case of false positives, it can record and send that snippet of data \cite{kruzel_2018}. There was also an issue with Google Homes and their wake button that caused the Home to listen 24/7 \cite{burke_2017}. Outside of these accidental cases, it is still unknown what exactly these devices do as the company's privacy policy is not clear \cite{kruzel_2018}.

This new era of internet-connected devices will produce countless innovative products, but also many security flaws. We also don't know everything these devices are doing. At this point, it is crucial to establish not only a baseline network and power usage of IoT devices, but also research into for anomalies, unusual patterns, and excessive network/power usage in these devices.

In this paper, we will go over how we created a database of network packets and power info from various IoT devices that other researchers can analyze without having to log this data. Then, we analyzed various devices within the database and found interesting patterns. Before we directly define our scope, we will present some previous work and state what we have done to build upon them.

\section{Previous Work}
In this section, we will provide previous works that have similarly sought to build a data set of IOT network traffic and use it for analysis. Here, we will present the previous works individually. We will separately comment on how their work is different from ours and how it is useful to us in the Scope section\ref{Scope}.

\subsection{An Analysis of Home IoT Network Traffic and Behaviour}
The work and goals of \textit{An Analysis of Home IoT Network Traffic and Behaviour}~\cite{home_iot} are the most similar to the work and goals of this paper. The \textit{An Analysis of Home IoT Network Traffic and Behaviour} hopes to analyze IoT traffic in the home. The authors' set up an IoT testbed consisting of an air quality monitor, Amazon Echo, a few Apple devices, a smart hub, and a smart vacuum cleaner. After intercepting this traffic, they analyze each device. They go over how many bytes of data each device transmitted over the 22 days of network logging. They split this data up by protocol and by incoming/outgoing traffic.

\subsection{ProfilIoT}
The paper, \textit{ProfilIoT: A Machine Learning Approach for IoT Device Identification Based on Network Traffic Analysis}~\cite{Meidan:2017:PML:3019612.3019878} uses machine learning algorithms to classify IoT devices. The researchers of this paper collect traffic from 13 different IoT and non-IoT devices. The IOT devices include a baby monitor, motion sensor, printer, refrigerator, security camera, television, laptop, and smartphone. The  non-IoT devices include a PC and smartphone. We include non-IoT devices for comparison. All of these devices connect to a WiFi access point that would record their network traffic with wire shark.

The researchers classify the specific device in use such as a smartphone, baby monitor, or motion sensor based single sessions and multiple sessions. A single session is a 4-tuple formatted as (src IP, dest IP, src Port Number, dest Port Number) enriched with Alexa Rank~\cite{alexa} and GeoIP~\cite{maxmind}. When intercepting network traffic, they extract the information they need to form the four-tuple data type. A multiple session is a list of single sessions. The paper states that they can classify smartphones from just the "user agent" HTTP property with 100 percent accuracy, they can classify PCs from single sessions with a false positive rate of 0.003, and they classify IOT devices with 88.271 percent accuracy using multiple sessions

%\subsection{Detection of Unauthorized IoT Devices Using Machine Learning Techniques}
%In the same year that ProfilIoT was released, the same group of 7 released another paper called \textit{Detection of Unauthorized IoT Devices Using Machine Learning Techniques}~\cite{meidan_2017}. In this paper, the researchers continued their work of ProfilIoT not only to classify devices but to also, given a list of whitelisted devices sets, classify a device then determine if it is apart of that whitelist. They hope to apply this to enterprise networks to prevent any unauthorized devices from connecting to it. They use the same four tuples as in the ProfilIoT paper and a bunch of other features to classify by in a random forest model. They found that time to live minimum is a significant factor in classifying the devices among other features.

\subsection{Logging and Analysis of Internet of Things (IoT) Device Network Traffic and Power Consumption}
\label{frawleyPaper}
\textit{Logging and Analysis of Internet of Things (IoT) Device Network Traffic and Power Consumption}\cite{frawley_2018}, written by Ryan Frawley, was formed in conjunction with this paper. Frawley's paper and this paper were both directed by advisor Andrew Danowitz at Cal Poly. The two papers can be put together and seen as one whole project overseen by Professor Danowitz with the goal to build a large database containing ten months of power usage and network packets from 16 different IoT devices and analysis on various devices within that database. This whole project has been under work for four quarters at Cal Poly. Frawley began working on his paper for the first three-quarters of the project. We began working on this paper for the last three-quarters of this project.

Frawley's paper goes over the steps in setting up an access point, connecting devices to the access point, logging the power information for these devices, logging the network information for these devices, setting up a sustainable IOT test bed, and analyzing these devices further. He looked at the GeoIP\cite{maxmind} and the packets of any unencrypted data in the devices.

He focuses on creating an access point that would network intercept packets and push them to a database. The access point also queries smart plugs connected to each IoT device for their power usage each second and records it to the same database in separate tables called 'ip\_log' and 'power' correspondingly. An event log keeps track of what each device is doing so that specific use cases can provide context to the database. The event log is a Google Sheets file. For example, when anyone uses the Google Home for streaming music, they would specify so with the start time, stop time, the device in use (Google Home), the action (streaming music), and any special notes (e.g., lag or other devices light up). Example entries for the event log are shown below in figure \ref{tab:events}.

\section{Scope}
\label{Scope}
The first paper from the previous works section\cite{home_iot} most closely matches this paper. It has the same overall idea to collect network data and then uses it to analyze metadata surrounding the networks. In our paper, we do the same thing, but instead, we built a portable database consisting of 10 months of data rather than 22 days that we hope to make public. We also analyze more than the number of bytes sent by each device and looked deeper to see if the classification of devices are possible and how these devices operate over time while under operation.

The second paper from the previous works section\cite{Meidan:2017:PML:3019612.3019878} focuses more on the classification of devices from network traffic, which is we have done. However, instead of utilizing machine learning techniques as they do, we focus on manual analysis (looking for spikes in power usage, network traffic and the height of the spike, the length of the power spike, and other graphical heuristics).

In comparison to the first \cite{home_iot} and second paper \cite{Meidan:2017:PML:3019612.3019878} from the previous section, we have also added power usage over time as a data set. The two paper's mentioned only focus on network traffic. We also put a significant emphasis on creating an extensive database that others can access rather than a smaller set of data.

This paper is essentially a continuation of the third paper in the previous works section \cite{frawley_2018}. As stated in the subsection for Frawley's paper\ref{frawleyPaper}, the project spanned four school quarters at Cal Poly. Ryan started on the project the first quarter, then I joined the second quarter where we worked together for two quarters, then Frawley finished his paper, and finally I worked on this project for the last quarter independently.

In the first quarter, he worked on creating the software for the IOT test bed. Then we both put together the IOT test bed and interacted with the devices on a daily basis to simulate regular usage. Then we both analyzed various devices in tandem. Then we separately did more analysis on our own. This paper has much overlap with Frawley's paper in the setup for the testbed and analysis. The paper will state if he had already mentioned work we had both done to prevent repeated information but still state the basics so that the research is known. In comparison to Frawley's paper, this paper analyzes the device power lines more. This paper sums the power usage for multiple devices to show its possible to classify devices was possible from just one power graph.

This paper will not analyze every device within the IOT test bed. It will focus on a select few devices and provide examples of how this database and its analysis tool could be used.

\section{Contributions}
This paper intends to create a database of IOT network traffic and power data; future researchers can use this database for their research without having to set up an IOT test bed. This paper will present a way to classify what smart speaker is in use from just minutes of power information from the database. We then introduce noise into this visual classification process and provide a comparison between each smart speakers network/power usage.

We want to determine if someone could identify what device a household is using with access to the household's power line. We prove the feasibility of this through classifying a summation of power over time for multiple devices. We choose to use this as our hypotheses because power monitors are on the outside of homes and self-installation monitoring tools are out there \cite{griffith_2017}. Nowadays, some power monitors are wireless, meaning it could be remotely hacked into \cite{griffith_2017}.

\section{Thesis Organization}
In chapter \ref{Method} this paper will discuss the steps to set up the IOT test bed, the analysis tool, and the logging system for interaction with devices. This chapter highlights any steps necessary to set up the scripts we have created. Chapter \ref{Results} will present power and network traffic for a few devices in the form of line graphs. Chapter \ref{Discussion} will discuss the data presented in chapter \ref{Results}, why a particular device might have higher throughput traffic and the feasibility of classifying devices within a household from just the outside power line. Finally, chapter \ref{Conclusion} finishes with concluding thoughts and future work.