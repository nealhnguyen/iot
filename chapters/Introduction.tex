\chapter{Introduction}
\label{Introduction}
There are 23.14 billion IoT devices in use worldwide that nubmer is expected to grow to 75.44 billion by 2025 \cite{statista_2016}. With so many manufacturers creating IoT devices, each with differing update policies, some devices inadvertently have better support. For example, some devices may drop out of a manufacturer's update cycle and become unsupported, introducing privacy and security concerns.

To address these concerns, this thesis contributes an IoT testbed that logs network and power data from 16 IoT devices over one year, accumulating 184.94 GB of data and 172,445,929 rows into a database. To help researchers sort and view this data, this thesis adds a Python program that graphs network traffic and power data over time from the database. The graphs created by this tool were used to analyze IoT devices network and power usage in the testbed while idle, during startup, and while in use. From these graphs, it is possible to identify the smart speaker in use when viewing just one minute of the shared power usage.

This paper focuses on addressing security and privacy flaws, that if fixed, do not affect the core features of an IoT device. For example, a smart speaker must store audio snippets to parse for the wake word. If the smart speaker occasionally hears a false positive wake word and sends the audio to its server, that is reasonable. However, Google Homes had an issue with their wake button, which caused the Home to listen 24/7 \cite{burke_2017}. This is not expected behavior and highlights privacy concerns. In another case, one of the largest IoT cybersecurity attacks, Mirai, was able to use weak login credentials to take control of 2.5 million IoT devices to perform a denial of service attack \cite{whittaker_2017}. This is on many events that introduces security concerns. This paper uses these flaws as a focus when analyzing network and power usage.

\section{Previous Work}
This section presents and analyzes related works on the topic of analyzing and characterizing IoT devices. It presents the previous works individually. Because these papers are similar to each other, commentary on how their work is different from ours and how it is useful to us as a group in section \ref{Scope}.

\subsection{An Analysis of Home IoT Network Traffic and Behaviour}
\label{homeIoTPaper}
The scope and work of \textit{An Analysis of Home IoT Network Traffic and Behaviour}~\cite{home_iot} are most like the work and goals of this paper. In \textit{An Analysis of Home IoT Network Traffic and Behaviour}~\cite{home_iot}, the authors analyze IoT traffic in the home. The authors created an IoT testbed by setting up multiple IoT devices, connecting them to a router, sniffing their network packets while idle, and storing these packets on a Linux box's disk. The IoT testbed consists of a smart air quality monitor, Amazon Echo, a few Apple devices, a smart hub, and a smart vacuum cleaner.

After 22 days of network logging, the authors analyzed each IoT device individually and as a whole. For example, they noticed that they can identify most devices from the first three MAC address bits. The Hue bridge broadcasts credentials over HTTP, which is unencrypted. The authors state that these seemingly small security flaws create a privacy risk. A user’s presence in a room or house can be determined from these unencrypted HTTP packets. The authors also show the percentage of network packets by protocol and various other device network patterns. This general analysis fingerprints each device.

\subsection{ProfilIoT}
\label{ProfilIoTPaper}
The paper, \textit{ProfilIoT: A Machine Learning Approach for IoT Device Identification Based on Network Traffic Analysis} ~\cite{Meidan:2017:PML:3019612.3019878} uses machine learning algorithms to classify IoT devices. The researchers of this paper collect traffic from 13 different IoT and non-IoT devices. The IOT devices include a baby monitor, motion sensor, printer, refrigerator, security camera, socket, thermostat, smartwatch, and television. The non-IoT devices include two PCs and two smartphones for comparison. These devices connect to a Wi-Fi access point that recorded their network traffic with Wire Shark\cite{wireshark}.

The researchers use machine learning on single-sessions to classify a device as an IoT device or non-IoT device. Then, they can classify the IoT devices by brand and model(e.g. Samsung Refrigerator, LG TV, WeMo Motion Sensor) with multi-sessions. A single-session is a 4-tuple formatted as source IP, destination IP, source port Number, destination port Number. When intercepting network traffic, they extracted the information they needed from each TCP packet to form the four-tuple data type. A multi-session is a list of single-sessions. Another machine learning model determines the minimum number of single-sessions needed to classify each device, determining the size of a multi-session. With single-sessions, they could determine if the device is an IoT device or not with 100 percent accuracy. Then out of their nine IoT devices, they can classify brand and model of the IoT device with 99.281 percent accuracy when run 7376 times.

\subsection{Logging and Analysis of Internet of Things (IoT) Device Network Traffic and Power Consumption}
\label{frawleyPaper}
\textit{Logging and Analysis of Internet of Things (IoT) Device Network Traffic and Power Consumption}\cite{frawley_2018}, written by Ryan Frawley, was formed in conjunction with this paper. Frawley's paper and this paper were both directed by advisor Andrew Danowitz at Cal Poly.

Frawley's paper documents the steps necessary to construct a reliable IoT testbed capable of capturing network traffic and power data for connected devices, and analyzing these devices further. He performed GeoIP\cite{maxmind} lookups on each device, showing the percentage of packets originating from each country and company. He also analyzed the packets of any unencrypted data in the devices.

\section{Scope}
\label{Scope}
The first paper from subsection \ref{homeIoTPaper} most closely matches this paper. The authors have the same overall idea to collect network data and then use it to analyze metadata surrounding the networks. Our work expands on this concept by contributing a portable database consisting of 10 months of data rather than 22 days of data. This paper adds more devices in our study and focuses more on device power/network usage over time rather than specific network packet information.

Then, like the second paper from subsection \ref{ProfilIoTPaper}, this paper also focuses on classification of devices from data. However, instead of using machine learning techniques on network data, this paper focuses on manual analysis, looking for spikes in power usage, the height of the spike, the length of the power spike, and other graphical heuristics.

In comparison to the first paper in \ref{homeIoTPaper} and second paper in \ref{ProfilIoTPaper}, this paper adds power usage over time to the data set. The two papers mentioned only focus on network traffic. This paper also puts a significant emphasis on creating an extensive database rather than a smaller set of data to create graphs on network and power usage over time.

This paper is a continuation of the third paper in subsection \ref{frawleyPaper}. Due to overlap between these two works, certain aspects of the IoT testbed setup and usage is only covered in cursory detail here. The reader is advised to access Frawley's work for full information. We both assembled the IOT test bed and interacted with the devices on a daily basis to simulate regular usage. We both also performed a preliminary analysis of the device network and power usage together.

The unique contribution of this work is its analysis of IoT device power usage and the introduction of a custom data visualization tool that attaches to the database. This paper focuses on a select few devices, analyzing their startup, idle, and in use network and power usage over time. I compare the smart speakers' network and power usage and show that it is possible to identify a smart speaker through analysis of the powerline over time.

\section{Thesis Organization}
Chapter \ref{Method} discusses my steps in setting up the IOT test bed, the analysis tool, and the logging system for interaction with devices. Chapter \ref{Method} also highlights the steps to set up a developer environment to run the analysis tool and how to use it. Chapter \ref{Results} presents power and network traffic for smart speakers and streaming devices while idle, in use, and during startup in the form of line graphs. Chapter \ref{Results} also shows the graphs used to fingerprint the smart speakers while handling different commands and under noise. Chapter \ref{Discussion} discusses the data presented in Chapter \ref{Results}, why a device might have higher throughput traffic and the feasibility of classifying devices within a household from a shared power line. Finally, Chapter \ref{Conclusion} finishes with concluding thoughts and future work.
