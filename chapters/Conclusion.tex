\chapter{Conclusion}
\label{Conclusion}
To conclude, this paper documents the creation of a large database of IoT devices' network and power usage, introduces a graphing tool for this database and shows that it is possible to tell what smart speaker is in use based on just the initial spike made during a command.

After one year of logging common household IoT devices' network and power usage, our AWS database contains 172,445,929 entries that take up 184.94 GB of data. Data is often overlooked as a crucial part of any project, such as research projects or AI projects. The process of obtaining it can be tedious. This database provides a large dataset for others to use without having to repeat the steps to form a testbed and log network traffic and power usage. This database was incredibly useful in Frawley and my work. Other students have already used it for their research projects.

Complementary to this database, a lot of data visualization was necessary to understand and extract patterns in IoT devices. To further streamline IoT research, a visualizer tool that graphs power and network usage for any device in the database makes visual analysis quick and easy. The visualizer provides many useful features that Frawley and I used for our research.

Some patterns can be difficult to make out with so much noise coming from within a house for power data. However, in some cases, it is still possible to determine what smart speaker is in use and whether the lights are on in the house from the power data. With more research, it is likely that other private information can be inferred from the power line data as the research here is highly informal and has yet to leverage machine learning to search for patterns. The results show a privacy flaw in household power lines.

Together, this proves that more research should be done to analyze and classify IoT devices. The latter analysis in smart speaker classification is informal, utilizing visual analysis and some numbers to strongly support our claims. A more formal study into device classification from a shared powerline can be done. Also, analysis with more than just smart speakers and power spikes can be done.

\section{Future Work}

With an ever-growing database, its maintenance is necessary. We want to clean up the database by indexing all columns for quick lookup, as of now, only some columns are indexed. There are also some inconsistencies in the database because an IP address of devices and naming of WeMos change over time. Changing the IP addresses and WeMo names that refer to the same device is necessary.

The next plan is to improve the realtimeIoTgrapher and add features. Hosting it on a public domain would make this tool more easily accessible, rather than running it locally and going to a local domain. Including more automated annotation on the graphs would also make anomaly detection easier.

We also want to utilize both power and network data in tandem for analysis to see if even more information can be extracted.

With many botnet attacks in the recent past, we also attempted to hack into these devices with Mirai and analyze power usage and network throughput of these devices. Getting Mirai working in the restricted time turned out to be unfeasible, but would be interesting to continue for future work.

Finally, applying machine learning to the power data would be interesting to pursue. We quickly tried a feed-forward neural net with 0 percent accuracy and an LSTM and achieved 30 percent accuracy given five devices (slightly better than random). We had also planned to alter a human activity recognition machine learning (HAR) model to fit the database for classification of device model given power data over a set period. It is likely that machine learning can find patterns that visual examination could not.

Regardless, there are many opened opportunities for future work because of the database. There have already been students who have used the database for various research projects. Researchers can quickly utilize the realtimeIoTGrapher to analyze the database or manually query data. The world of IoT needs careful analysis and research, and hopefully, my research reduces some friction for more projects about IoT devices.