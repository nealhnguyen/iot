\chapter{Conclusion}
\label{Conclusion}

To conclude, this paper shows that it is easy to tell which smart speaker is in use based on just the initial spike. Devices are each making their unique trade-off between power and network throughout when idle and in use. This tradeoff leads to unique signatures that allow us to classify devices. For the smart speakers, it is as obvious as the first power spike during usage.

Some patterns can be difficult to make out with so much noise coming from within a house for power data. However, if enough devices are idle, it is still possible. The results show an indirect security flaw in our powerline and how that defense should be fortified. This paper shows it is possible to determine what smart speaker is in use and whether the lights are on in the house from the power data. With more research, it is likely that other private information can be inferred from the power line data.

This proves that more research should be done to analyze and classify IoT devices. We created an expansive database of network and power data for 16 IoT devices that spread over ten months, accumulating 184.94 GB of data and 172,445,929 entries in the database. Along with that, we created an analysis tool to examine the database visually.

\section{Future Work}

With an ever-growing database, its maintenance is necessary. We want to clean up the database by indexing all columns for quick lookup, as of now, only some columns are indexed. There are also some inconsistencies in the database because an IP address of devices and naming of WeMos change over time. Changing the IP addresses and WeMo names that refer to the same device is necessary.

The next plan is to improve the realtimeIoTgrapher and add features. Hosting it on a public domain would make this tool more easily accessible, rather than running it locally and going to a local domain. Including more automated annotation on the graphs would also make anomoly detection easier.

We want to also utilize both power and network data in tandem for analysis to see if that yields better classification results.

With many botnet attacks in the recent past, we also attempted to hack into these devices with Mirai and analyze power usage and network throughput of these devices. Getting Mirai working in the restricted time turned out to be unfeasible, but would be interesting to continue for future work.

Finally, applying machine learning to the power data would be novel work. We quickly tried a feed forward neural net with 0 percent accuracy and an LSTM and achieved 30 percent accuracy given five devices (slightly better than random). We had also planned to alter a human activity recognition machine learning (HAR) model to fit the database for classification of device model given power data over a set period. Due to time restraints, I did not explore this further.

Regardless, there are many opened opportunities for future work because of the database. Researchers can quickly utilize the realtimeIoTGrapher to analyze the database or manually query data. There have already been students who have used the database for various research projects.