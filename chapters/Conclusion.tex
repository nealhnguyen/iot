\chapter{Conclusion}
\label{Conclusion}
To conclude, this paper documents the creation of a large database of IoT devices' network and power usage, introduces a graphing tool for this database and shows that it is possible to tell what smart speaker is in use based on just the initial spike made during a command.

After one year of logging common household IoT devices' network and power usage, our AWS database contains 172,445,929 entries that take up 184.94 GB of data. This database provides a large dataset for others to use without having to repeat the steps to form a testbed. This database was incredibly useful in Frawley and my work. Other students have already used it for their research projects.

To further streamline IoT research, a visualizer tool was created that graphs power and network usage for any device in the database, making visual analysis quick and easy. The visualizer provides many useful features that Frawley and I used for our research.

Some patterns can be difficult to make out with so much noise coming from within a house for power data. However, in some cases, it is still likely possible to determine what smart speaker is in use. With more research, it is likely that other private information can be inferred from the power line data as the research here has yet to leverage advanced statistical methods.

\section{Future Work}
With an ever-growing database, maintenance is necessary. We want to clean up the database by indexing all columns for quick lookup; as of now, only some columns are indexed. There are also some inconsistencies in the database because the IP addresses of the devices and WeMo names change over time. Unifying a device under one IP address and a WeMo under one name in the database are clear first steps.

The next plan is to improve the realtimeIoTgrapher and add features. The biggest interest is to add automatic annotations such as boxes denoting peak to peak spikes, or trend line plots. Plotly also has a feature to display the grapher on a public domain. Altering the grapher program to do so would also make this tool more easily accessible as compared to the locally hosted setup shown in this paper.

We also want to utilize both power and network data in tandem for analysis to see if even more information can be extracted.

With many botnet attacks in the recent past, we would like to hack into these devices with Mirai and analyze power usage and network throughput of these devices. This paper shows trends between power use, network traffic, and what an IoT device is doing. It is possible that power use and network traffic could correlate with Mirai's operations. Getting Mirai working in the restricted time turned out to be infeasible, but would be interesting to continue for future work.

Finally, applying advanced statistical methods to the power data would be interesting to pursue. We quickly tried machine learning with a feed-forward neural net with 0 percent accuracy and an LSTM and achieved 30 percent accuracy given five devices (slightly better than random). We had also planned to alter a human activity recognition machine learning (HAR) model to fit the database for classification of device model given power data over a set period. It is likely that machine learning can find patterns that visual examination could not.

Regardless, there are many opened opportunities for future work because of the database. There is already a student who is using the database for their research project. Researchers can quickly use the realtimeIoTGrapher to analyze the database or manually query data.

The world of IoT needs careful analysis and research for privacy and security. Hopefully, my contributions will help enable these studies.